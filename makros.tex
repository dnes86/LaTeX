%%%%%%%%%%%%%%%%%%%%%%%%%%%%%%%%%%%%%%%%%%%%%%%%%%%%%%%%%%%%%%%%%%%%%%%%
% Thesis:	
% Thema:	
% Name:		
% e-Mail:	
%%%%%%%%%%%%%%%%%%%%%%%%%%%%%%%%%%%%%%%%%%%%%%%%%%%%%%%%%%%%%%%%%%%%%%%%
%
%-----------------------------------------------------------------------
%%%%%%%%%%%%%%%%%%%%%%%%%%%%%%%%%%%%%%%%%%%%%%%%%%%%%%%%%%%%%%%%%%%%%%%%
%%%% BEGINN MAKROS %%%%%%%%%%%%%%%%%%%%%%%%%%%%%%%%%%%%%%%%%%%%%%%%%%%%%
%%%%%%%%%%%%%%%%%%%%%%%%%%%%%%%%%%%%%%%%%%%%%%%%%%%%%%%%%%%%%%%%%%%%%%%%

%-----------------------------------------------------------------------
% Definieren der Dokumentklasse / Papierformat 
%-----------------------------------------------------------------------
\documentclass[			% Standard-Dokument mit
		a4paper,		% Papierformat A4 (210x297mm)
		12pt,			% Schrift 12-Punkt
		twoside,		% zweiseitig 
		BCOR12mm,		% Bindekorrektur, bei zweiseitigen der innere Rand
		DIV12,			% Seitenränder
		headsepline,		% Trennlinie nach der Kopfzeile
		cleardoubleempty,	% Kopf- und Fusszeilen bei leeren Seiten ausblenden
		chapterprefix,		% vor der Kapitelnummer in der Überschrift das Wort 'Kapitel' setzen 
		parskip,		% bestimmt die Größe eines Absatzes
		liststotoc,		% Verzeichnisse ins Inhaltsverzeichnis aufnehmen
	     % idxtotoc,		% Index ins Inhaltsverzeichnis aufnehmen
		bibtotoc,		% Literaturverzeichnis ins Inhaltsverzeichnis aufnehmen
		final			% Status des Dokuments (final/draft)
					% draft: Overfull Boxes anzeigen & keine Grafiken einbinden
	      ]{scrbook}	% Dokumentklasse 'scrbook' für die Abschlussarbeit
			

%-----------------------------------------------------------------------
%\usepackage[iso]{umlaute}			% Zeichencodierung											
%\usepackage[latin1]{inputenc}			% für UNIX und auch Windows-Systeme
%\usepackage[ansinew]{inputenc}			% für Windows-Systeme
%\usepackage[applemac]{inputenc} 		% für MacOSX-Systeme
\usepackage[utf8]{inputenc} 			% für Linux-Systeme
%\usepackage[utf8x]{inputenc} 			% ist eine Alternative für alle obigen Systeme

\usepackage[T1]{fontenc}			% Korrektes Trennen von Wörtern mit Umlauten, Sonderzeichen, etc.
\usepackage[english, ngerman]{babel}		% english & deutsch, neue Rechtschreibung bzw. neue Silbentrennung
\usepackage[german]{varioref} 			% Referenzierung mit variablen Texten
\usepackage{bibgerm} 				% um ein deutschsprachiges Literaturverzeichnis zu haben

\usepackage[scaled=0.92]{helvet}
\usepackage{courier}				% Schriftart "Courier"
\usepackage{textcomp}	 			% Zusatzliche Symbole (Euro-Zeichen etc.)

\usepackage{color, xcolor, colortbl}		% Erweitertes Farbpaket mit vielen Farbmodellen
\usepackage{graphicx}				% Grafiken bzw. Bilder einbinden
\usepackage{subfigure} 				% mehrere Abbildungen nebeneinander/übereinander
\graphicspath{{figures/}}			% Festlegung des Pfades zu den Bildern
\DeclareGraphicsExtensions{.png,.pdf,.eps}	% Dateiendung für Bilder
\usepackage{pdfpages}                           % PDF Seiten einfügen
\usepackage{grffile}
\usepackage{here}                               % Bild an einer bestimmten stelle zu haben [H]

\usepackage{setspace}		% ermöglicht ein einfaches Umstellen von Zeilenabstand 1.0, 1.5 oder 2
\usepackage[plainpages=false,pdfpagelabels=true]{hyperref} % anklickbares Inhaltsverzeichnis
\usepackage{cite}
\usepackage{paralist}           % für die compactenum Aufzählung
\usepackage{enumitem}

%\usepackage{makeidx}		% Indexerstellung
%\usepackage{nomencl}		% ist ein Makro, welches mit Hilfe von MakeIndex ein Abkürzungsverzeichnis erstellt

%\setcapindent{1em}		% KOMA-Script Option, Zeilenumbruch bei Bildbeschreibungen
\usepackage{scrhack}		% Paket 'scrhack' verbessert neben hyperref auch andere Pakete und kann
				  % auch mit anderen Klassen als den KOMA-Script-Klassen verwendet werden.
																			
%\usepackage{lscape}		% z.B. für Tabellen oder Abbildungen
%\usepackage{pdflscape}		% Rotieren von Seiten 
%\usepackage{rotating}		% Rotieren von Text, Tabellen
%\usepackage{a4wide} 		% ganze A4 Weite verwenden



%-----------------------------------------------------------------------
% Befehle für Zeichen[ketten]
%-----------------------------------------------------------------------
\newcommand{\todotext}[1]{{\color{red} TODO: #1} \normalfont}	% Befehl:  \todotext{text}
\newcommand{\abs}[1]{\textbf{#1:}}
\newcommand{\CLI}[1]{\texttt{#1}} % Schreibmaschinenschrift (command-line interface / Konsole / Terminal) 
%\hyphenpenalty=10000		  % Silbentrennung bis zum Dokumentenende ausschalten
\usepackage{blindtext}		  % Platzhaltertext generieren 



%-----------------------------------------------------------------------
% Befehle für Tabellen
%-----------------------------------------------------------------------
\usepackage{tabularx}		% Tabellen haben eine feste Gesamtbreite haben und eine variable Spaltenbreite
%\usepackage{ltxtable}		% Tabellen haben keine feste Gesamtbreite haben und keine variable Spaltenbreite
\usepackage{multirow} 		% Tabellen-Zellen über mehrere Zeilen
\usepackage{multicol} 		% mehre Spalten auf eine Seite
\usepackage{booktabs} 		% um korrekt gesetzte, Schöne Tabellen zu erzeugen mit \toprule, \midrule, \bottomrule
%\usepackage{longtable} 	% Für Tabellen, die länger als eine Seite sind. Seitenumbruch erfolgt automatisch.
%\usepackage{array}		% Paket für erweiterte Tabelleneigenschaften
%\usepackage{float}		% Paket unterstützt bei der Positionierung von Grafiken und Tabellen



%-----------------------------------------------------------------------
% Befehle für Mathematiksatz
%-----------------------------------------------------------------------
\usepackage{amsmath}		% Es wird das AMS-Paket verwendet
\makeatletter
\def\displaymath{\typeout{*** Do not use DISPLAYMATH. Use \[...\] instead ***}}
\def\eqnarray{\typeout{*** Do not use EQNARRAY(*). Use align(*)-environment instead ***}}
\makeatother
\usepackage{amsfonts}		% math fonts
\usepackage{amstext}
\usepackage{mathptmx}
%\usepackage{calc}		% Rechnen in TeX
%\usepackage{dsfont}
%\usepackage{mathtools}
%\usepackage{latexsym}
%\usepackage{amsthm} 		% AMS-Pakete fuer Theoreme
%\usepackage{amscd}
\usepackage{amssymb}		% Häkchen mit \checkmark zum Abhaken
%\usepackage{framed} 



%-----------------------------------------------------------------------
% Einstellungen des Outputs des PDF-Dokuments	
%-----------------------------------------------------------------------
\usepackage{ifpdf}

\pdfcompresslevel=9		% beste Kompressionlevel
\pdfoutput=1			% ist n > 0 wird eine pdf-Datei erzeugt
				% ist n = 0 wird eine dvi-Datei erzeugt
	
\pdfpkresolution=2400		% für Drucker mit 2400 dpi (Standard: 600 dpi)

\ifpdf
    \definecolor{brown}{cmyk}{0, 0.81, 1, 0.60}
	\hypersetup{
%		pdftex=true,			%
%		draft=true,			% hyperref-Paket abschalten
%		backref=true,			% Rücklinks von bibitem, erfordert Leerzeile nach jedem bibitem
%		pagebackref=true,
		breaklinks=true,		% Zeilenumbruch bei Links
		plainpages=false,		% Fehlermeldungen reduzieren
%
	%---------------------------
	% Adobe Reader Anzeigeoptionen
	%---------------------------
%		bookmarks=true,            	% Acrobat-Lesezeichen einfügen
%   		bookmarksopen=true,        	% Acrobat-Lesezeichen links geöffnet
		bookmarksopenlevel=section,	% Geöffnete Hierarchiestufe (section OR 1)
   		bookmarksnumbered=true,  	% Kapitelnummern für Lesezeichen
		%	
 		pdfstartview=FitV, 		% PDF-Viewer benutzt beim Start bestimmte Seitenbreite (Fit, FitH, FitV, FitR, FitB, FitBH)
		pdfpagelayout=OneColumn,	% Seitendarstellung (SinglePage, OneColumn, TwoPageLeft, TwoPageRight)
		pdfpagemode=UseOutlines,	% Verhalten beim Öffnen des PDF:
						  %  UseOutlines  Zeige Lesezeichen (Standard)
                                   	  	  %  UseThumbs    Zeige Seitenvorschau
                                   	  	  %  FullScreen   Vollbid
%		
		pdflang=de,			% Sprache des Dokuments
		pdfdisplaydoctitle=true,	% Dokumenttitel statt Dateiname anzeigen
		pdfstartpage={1},       	% Seite, die angezeigt wird
		pdfhighlight={/O},		% Effekt beim selektieren eines Linktextes (O: Umrandung des Textes | N, I, O, P)
%		pdftoolbar=false,     		% Acrobat’s toolbar (true/false)
%		pdfmenubar=false,        	% Acrobat’s menu (true/false)
%		pdffitwindow=false,     	% PDF ins Fenster einpassen
%		pdfnewwindow=true,		%
%
	%---------------------------
	% Diese Farbdefinitionen zeichnen Links im PDF farblich aus
	%--------------------------- Notiz: für den Druck sollte am besten alles schwarz verwendet werden
		colorlinks=true,   		% aktiviert farbige Referenzen / true = farbige Links, false = Rahmen
   		linkcolor=black,		% Farbe für Links auf gleicher Seite (Notiz: black OR {0 0 0})
    		citecolor=brown,		% Farbe für Links auf Zitatstellen
		filecolor=magenta,		% Farbe für Links auf Dateien (Verknüpfungen, die lokale Dateien öffnen)
		urlcolor=blue,			% Farbe für Links (nicht nur URLs) / black, cyan, magenta, etc.
	  	anchorcolor=black,		% Farbe für Anker
		menucolor=red,             	% Farbe für Acrobatmenüeinträge
%		linkbordercolor={1 0 0},			
%		citebordercolor={0 1 0},
%		menubordercolor={1 1 0},
%		pagecolor=red,
%
	%---------------------------
	% PDF-Datei-Informationen
	%---------------------------
	  	baseurl={http://www.URL.de},			% URL			
		pdftitle={TITEL},                               % Titel des PDF Dokuments
		pdfsubject={THEMA},				% Thema des PDF Dokuments
  		pdfauthor={NAME},				% Autor des PDF Dokuments
  		pdfcreator={pdflatex},				% Erzeuger des PDF Dokuments
		pdfproducer={LaTeX with hyperref},	
  		pdfkeywords={STICHWÖRTER}		        % werden von Suchmaschinen auch für PDF Dokumente indexiert
	}			
\else   
\fi


%--------------------------------------------------------------------------------
% erstelle eine liste der verwendeten Klassen, Paketen, Dateien und deren Version
%--------------------------------------------------------------------------------
\listfiles	
